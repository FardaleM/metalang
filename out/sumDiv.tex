\documentclass[8pt]{article}
\usepackage[ruled,vlined]{algorithm2e}
\title{sumDiv}
\begin{document}
\maketitle
\begin{algorithm}[H]
\SetKwInOut{Input}{input}
\SetKwInOut{Output}{output}
\Input{}
\Output{void}

$int\:a := 0$\;
\tcc*[f]{\emph{ test }}\;
a ++;
\tcc*[f]{\emph{ test 2 }}\;
\caption{foo}
\end{algorithm}
\begin{algorithm}[H]
\SetKwInOut{Input}{input}
\SetKwInOut{Output}{output}
\Input{}
\Output{void}

\caption{foo2}
\end{algorithm}
\begin{algorithm}[H]
\SetKwInOut{Input}{input}
\SetKwInOut{Output}{output}
\Input{}
\Output{void}

\If{ $ 1 = 1 $ }{
  
}

\caption{foo3}
\end{algorithm}
\begin{algorithm}[H]
\SetKwInOut{Input}{input}
\SetKwInOut{Output}{output}
\Input{n:int}
\Output{int}
\tcc*[f]{\emph{ On d\'esire renvoyer la somme des diviseurs }}\;
$int\:out := 0$\;
\tcc*[f]{\emph{ On d\'eclare un entier qui contiendra la somme }}\;
\For{ $i\;:=\;1\; to\;n$}{
                            \tcc*[f]{\emph{ La boucle : i est le diviseur potentiel}}\;
                            \If{ $ (n mod i) = 0 $ }{
                              \tcc*[f]{\emph{ Si i divise }}\;
                              out += $ i $;
                              \tcc*[f]{\emph{ On incr\'emente }}\;
                            }
                            \Else{
                              \tcc*[f]{\emph{ nop }}\;
                            }}

\Return $ out $\;
\tcc*[f]{\emph{On renvoie out}}\;
\caption{sumdiv}
\end{algorithm}

\begin{algorithm}[H]
\tcc*[f]{\emph{ Programme principal }}\;
$int\:n := 0$\;
$read_{int}$(n)\;
\tcc*[f]{\emph{ Lecture de l'entier }}\;
$print_{int}$($ sumdiv(n) $)\;
\caption{Main}
\end{algorithm}
\end{document}

